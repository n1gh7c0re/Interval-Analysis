\documentclass[12pt,a4paper]{article}

% Russian support
\usepackage[T2A]{fontenc}
\usepackage[utf8]{inputenc}
\usepackage[russian]{babel}

% Math, URL, layout
\usepackage{amsmath,amssymb,amsthm}
\usepackage{geometry}
\usepackage{hyperref}
\usepackage{microtype}
\usepackage{setspace}
\usepackage{caption}
\usepackage{graphicx} % для вставки графиков / заглушек
\usepackage{booktabs} % для таблиц (если понадобятся)
\geometry{
  left=25mm,
  right=25mm,
  top=25mm,
  bottom=25mm
}

\hypersetup{
  colorlinks=true,
  linkcolor=black,
  urlcolor=blue,
  pdfauthor={Тишковец Сергей},
  pdftitle={Отчёт по лабораторной работе №2 — Интервальный анализ}
}

\begin{document}
\selectlanguage{russian}
\begin{titlepage}
  \begin{center}
    {Санкт-Петербургский политехнический университет Петра Великого\\
    Институт прикладной математики и механики\\[0.5em]
    \textbf{Высшая школа прикладной математики и вычислительной физики}}
    
    \vfill
    
    {\LARGE\bfseries Отчёт по лабораторной работе №2\par}
    \vspace{1em}
    {\Large по дисциплине\\[0.3em] {\bfseries «Интервальный анализ»}}
    
    \vfill
    
    \begin{flushright}
        Выполнил\\
        студент гр. 5030102/20202: Тишковец Сергей
    \end{flushright}
    
    
    \begin{flushright}
        Проверил\\
        Преподаватель: Баженов Александр Николаевич
    \end{flushright}
    
    \vfill
    Санкт-Петербург\\
    2025
  \end{center}
\end{titlepage}

\setcounter{page}{2}

\tableofcontents
\newpage

\section{Цель работы}

Исследование и сравнение точности различных интервальных методов оценивания области значений функции на заданном отрезке.

\section{Постановка задачи}

Для каждой из двух функций \(f_1(x)\) и \(f_2(x)\) на интервале \(X=[a,b]\) необходимо:

\begin{enumerate}
\item Аналитически или численно найти область значений \(\operatorname{ran}(f,X)\), построить график функции на заданном интервале.
\item Вычислить интервальные оценки области значений, используя:
\begin{enumerate} 
  \item Естественное интервальное расширение исходного выражения функции.
  \item Естественное интервальное расширение эквивалентного выражения функции, полученного с помощью схемы Горнера или иного алгебраического преобразования.
  \item Дифференциальную центрированную форму с центром в разных точках интервала.
  \item Наклонную центрированную форму с центром в разных точках интервала.
  \item Бицентрированную форму.
\end{enumerate}

\item Для каждой полученной интервальной оценки вычислить величину \(\operatorname{dist}(F(X),\operatorname{ran}(f,X))\) — расстояние по Хаусдорфу до точной области значений. Проанализировать точность естественного интервального расширения:
\begin{enumerate}
  \item Найти (аналитически или численно) константу Липшица \(L\) для функции \(f\) на интервале \(X\).
  \item Получить теоретическую оценку погрешности \(\operatorname{rad}(F(X)) \le L\cdot\operatorname{rad}(X)\).
  \item Сравнить реальную погрешность (полуширину полученного интервала \(\operatorname{rad}(F(X))\)) с теоретической оценкой.
\end{enumerate}

\item Сравнить и проанализировать результаты, объяснив наблюдаемую точность или неточность каждого метода.
\end{enumerate}

\newpage

\section{Результат}

Функции:

\[
f_1(x) = x^3 - 3x^2 + 2,\quad X_1=[0,3]
\]

\[
{\,f_2(x) = x^5 - 5x + \sin x,\quad X_2=[-1.5,\,1.5]\,}
\]

\subsection{Анализ и точные области значений}

\paragraph{Для \(f_1\).}
Производная:
\[
f_1'(x)=3x^2 - 6x = 3x(x-2).
\]
Критические точки: \(x=0,\,2\). На отрезке \([0,3]\):
\[
f_1(0)=2,\quad f_1(2)=-2,\quad f_1(3)=2.
\]
Таким образом, точная область значений:
\[
\operatorname{ran}(f_1,[0,3])=[-2,\,2].
\]

\begin{figure}[h!]
  \centering
  % если файл в той же папке, что main.tex:
  \includegraphics[width=0.8\linewidth]{f1.png}
  \caption{График $f_1(x)=x^3-3x^2+2$ на отрезке [0,3].}
  \label{fig:f1}
\end{figure}

\paragraph{Для \(f_2\).}
Производная:
\[
f_2'(x) = 5x^4 - 5 + \cos x.
\]
Численное решение уравнения \(f_2'(x)=0\) (при наборе начальных приближений) дало приближённые критические точки:
\[
[-0.97,\ -0.083,\ 0.083,\ 0.97].
\]
Точки перегиба (по второй производной) — приблизительно \([-0.223,\,0.223]\).
Отрезок \([-1.5,1.5]\) включает все найденные критические точки и точки перегиба, поэтому на нём содержатся глобальные экстремумы/перегибы для изучаемой области.

Численные оценки точной области значений (получены в результате вычислений):
\[
\operatorname{ran}(f_2,[-1.5,1.5]) = [-3.16,\;3.16].
\]

\begin{figure}[h!]
  \centering
  % если файл в той же папке, что main.tex:
  \includegraphics[width=0.8\linewidth]{f2.png}
  \caption{График $f_2(x) = x^5 - 5x + \sin x$ на отрезке [-1.5,1.5].}
  \label{fig:f1}
\end{figure}

\subsection{Интервальные оценки (методы)}

\subsubsection{Естественное интервальное расширение}

Естественное интервальное расширение (natural interval extension) — это метод, при котором каждая переменная $x$ в выражении заменяется интервалом $X = [a, b]$, а все арифметические операции выполняются в интервальной арифметике.

\medskip

\textbf{Для $f_1$:}
\[
f_1(X) = X^3 - 3X^2 + 2 
       = [0,3]^3 - 3[0,3]^2 + 2
       = [0,27] - 3[0,9] + 2
       = [-25,29].
\]

Полученная оценка:
\[
\boxed{F_{\text{nat}}(X)=[-25,\,29]}
\]

\medskip

\textbf{Для $f_2$:}
\[
\begin{aligned}
f_2(X) &= X^5 - 5X + \sin X  =\\[6pt]
       &= [-1.5,\,1.5]^5 \;-\; 5[-1.5,\,1.5] \;+\; \sin([-1.5,\,1.5]) =\\[6pt]
       &= [-7.59,\,7.59] + [-7.5,\,7.5] + [-1,\,1] =[-16.09,\,16.09].
\end{aligned}
\]

\medskip

Полученная оценка:
\[
\boxed{F_{\text{nat}}(X) = [-16.09,\;16.09]}
\]

\subsubsection{Схема Горнера}
Схема Горнера позволяет представить многочлен в виде, который уменьшает количество повторяющихся переменных, тем самым снижая эффект зависимости (dependency effect).

\medskip

\textbf{Для $f_1$:} 

\medskip

Представим $f_1(X) = X^3 - 3X^2 + 2$ в виде:
\[
f_1(x) = (x-3)x^2 + 2
       = ([0,3]-3)[0,3]^2 + 2
       = [-3,0]\cdot[0,9] + 2
       = [-25,2].
\]

\medskip

Полученная оценка:
\[
\boxed{F_{\text{Horner}}(X) = [-25,\,2]}
\]
\medskip

\textbf{Для $f_2$:} 

\medskip

Представим $f_2(X)=x^5-5x+\sin x$ в виде:
\[
\begin{aligned}
f_2(x) &= x(x^4-5) + \sin x =\\[4pt]
       &= [-1.5,\,1.5]\cdot\big( [-1.5,\,1.5]^4 - 5\big) + \sin([-1.5,\,1.5]) =\\[4pt]
       &= [-1.5,\,1.5]\cdot\big( [0,\,5.0625] - 5\big) + [-\sin(1.5),\,\sin(1.5)] =\\[4pt]
       &= [-1.5,\,1.5]\cdot[-5,\,0.0625] + [-0.99,\,0.99] =\\[4pt]
       &= [-7.5,\,7.5] + [-0.99,\,0.99] = [-8.49,\,8.49].
\end{aligned}
\]

\medskip

Полученная оценка:
\[
\boxed{F_{\text{Horner}}(X) = [-8.49,\,8.49]}
\]

\subsubsection{Дифференциальная центрированная форма (DCF)}
Дифференциальная центрированная форма (DCF) использует разложение функции в ряд Тейлора первого порядка.

\[
f(X)=f(c)+f'(X)(X-c),\qquad\text{где }c\text{ — центр интервала }X.
\]

\medskip

\textbf{Для $f_1$:}  Пусть $c=1.5$, $f_1(1.5)=-1.375$.

\[
f_1'(x)=3x^2-6x.
\]

На интервале $[0,3]$ функция $f_1'(x)$ имеет минимум в точке $x=1$ (потому что $f_1''(x)=6x-6$ и $6x-6=0$ при $x=1$) и максимум в точке $x=3$.

\[
f_1'(1)=3\cdot1^2-6\cdot1=-3,\qquad
f_1'(3)=3\cdot3^2-6\cdot3=9,
\]

откуда
\[
f_1'(X)=[-3,\,9].
\]
\[
f_1(X)= -1.375 + [-3,\,9]\cdot[-1.5,\,1.5] = -1.375 + [-13.5,\,13.5] = [-14.875,\,12.125].
\]

\medskip

Полученная оценка:
\[
\boxed{F_{\mathrm{DCF}}(X)=[-14.875,\,12.125]}
\]

\medskip

\textbf{Для $f_2$:} Пусть $c=0$. Тогда
\[
f_2(0)=0^5 - 5\cdot 0 + \sin 0 = 0.
\]

Производная:
\[
f_2'(x) = 5x^4 - 5 + \cos x.
\]

На интервале $[-1.5,\,1.5]$ найдем оценки для $f_2'(x)$.

\[
f_2'(0) = -5 + 1 = -4.
\]

\[
f_2'(\pm 1) = 5 - 5 + \cos 1 \approx 0.5403.
\]

\[
f_2'(\pm 1.5)
= 5\cdot (1.5)^4 - 5 + \cos(1.5)
= 25.3125 - 5 + 0.0707
\approx 20.383.
\]

Следовательно,
\[
f_2'(X) = [-4,\; 20.383].
\]

Так как $c=0$, то
\[
X - c = [-1.5,\,1.5].
\]

Теперь вычислим:
\[
f_2(X)
= f_2(c) + f_2'(X)(X-c)
= 0 + [-4,\;20.383]\cdot[-1.5,\,1.5].
\]

Произведение интервалов:
\[
[-4,\;20.383]\cdot[-1.5,\,1.5]
= [-30.57,\; 30.57].
\]

\medskip

Полученная оценка:
\[
\boxed{F_{\mathrm{DCF}}(X) = [-30.57,\; 30.57]}
\]

\subsubsection{Наклонная центрированная форма (SCF)}
Наклонная центрированная форма (SCF) использует понятие наклонной функции (slope
function) $s(x,y)$, которая представляет собой интервал, содержащий все частные
производные в интервале $[x,y]$.

\[
f(X) = f(c) + S(X,c)(X-c), \qquad \text{где } S(X,c) \text{ — интервальная оценка наклонной функции}.
\]

\[
S(X,c) = \frac{f(X) - f(c)}{X - c}.
\]

\medskip

\textbf{Для $f_1$:} Пусть $c = 1.5$, $f_1(1.5) = -1.375$.

\[
S_1(x,c) = x^2 - 1.5x - 2.25
\]

На интервале $[0,3]$, $S_1(x)$ имеет минимум в точке $x = 0.75$ и максимум в точке $x = 3$:

\[
S_1(0.75) = 0.75^2 - 1.5 \cdot 0.75 - 2.25 = -2.8125, \qquad
S_1(3) = 3^2 - 1.5 \cdot 3 - 2.25 = 2.25
\]

\[
S_1(X,c) = [-2.8125,\; 2.25]
\]

\[
f_1(X) = -1.375 + [-2.8125,\; 2.25]\cdot[-1.5,\; 1.5] = [-5.594,\; 2.844].
\]

Полученная оценка:
\[
\boxed{F_{\mathrm{SCF}}(X) = [-5.594,\; 2.844]}
\]

\medskip

\textbf{Для $f_2$:} Пусть $c=0$. Тогда $f_2(0)=0$ и
\[
S_2(x,c)=\frac{f_2(x)-f_2(0)}{x-0}=\frac{x^5-5x+\sin x}{x}=x^4-5+\frac{\sin x}{x},
\]
где в точке $x=0$ значение берётся по пределу: $\displaystyle\lim_{x\to0}S_2(x,0)=-4$.

\medskip

Вычислим значения на стержневых точках интервала:
\[
S_2(0)=-4,
\qquad
S_2(1.5)=1.5^4-5+\frac{\sin(1.5)}{1.5}=5.0625-5+\frac{0.997}{1.5}\approx 0.727.
\]

Функция $S_2(x,0)$ чётная на отрезке $[-1.5,1.5]$, и её минимальное значение достигается в точке $x=0$, максимальное — в $x=1.5$. Следовательно
\[
S_2(X,0)=[-4,\;0.727].
\]

Так как $X-c=[-1.5,\,1.5]$, получаем
\[
f_2(X)=f_2(c)+S_2(X,0)\cdot (X-c)
=0 + [-4,\;0.727]\cdot[-1.5,\;1.5].
\]

Вычисление произведения интервалов даёт
\[
[-4,\;0.727]\cdot[-1.5,\;1.5]=[-6.01,\;6.01].
\]

Полученная оценка:
\[
\boxed{F_{\mathrm{SCF}}(X) = [-6.01,\;6.01]}
\]

% \subsubsection{Бицентрированная форма (BCF)}
% Бицентрированная форма (BCF) — это пересечение двух центрированных форм с разными центрами.

% \medskip

% \textbf{Для $f_1$:} 

% \medskip

% \textbf{Центр 1:} \(c_1 = 1.5\)
 
% \[
% F_1(X) = [-5.594,\, 2.844].
% \]

% \textbf{Центр 2:} \(c_2 = 2\) (критическая точка $f_1(x)$)
% \[
% f_1(2) = -2.
% \]
% \[
% X - c_2 = [0,3] - 2 = [-2,1].
% \]

% Наклонная функция:
% \[
% S(x,c_2) = x^2 - x - 2.
% \]

% Область значений $S_1(x,c_2)$ на $[0,3]$: производная
% \[
% s_1'(x) = 2x - 1 = 0 \quad \text{при } x = 0.5.
% \]

% Значения: 
% \[
% S_1(0) = -2,\qquad S_1(0.5) = -2.25,\qquad S_1(3) = 4.
% \]

% Следовательно,
% \[
% S_1(X) = [-2.25,\, 4].
% \]

% \[
% f_1(X) = -2 + [-2.25, 4]\cdot[-2, 1] = [-10,\, 2.5].
% \]

% \[
% F_{\mathrm{BCF}}(X) = [-5.594,\, 2.844] \;\cap\; [-10,\, 2.5] 
% = [-5.594,\, 2.5].
% \]

% Полученная оценка:
% \[
% \boxed{F_{\mathrm{BCF}}(X) = [-5.594,\; 2.5]}
% \]

% \medskip

% % BCF для f_2(x)=x^5-5x+\sin x на X=[-1.5,1.5], округление до 2 знаков
% \textbf{Для $f_2$:}

% \medskip

% \textbf{Центр 1:} \(c_1=0\)

% \[
% f_2(0)=0,\qquad S_1(x,0)=\frac{f_2(x)-f_2(0)}{x-0}=x^4-5+\frac{\sin x}{x},
% \]
% (по пределу \(S_1(0)=f_2'(0)=-4\)). Численным исследованием на \(X\) получаем
% \[
% S_1(X,0)=[-4,\;0.73],
% \qquad X-c_1=[-1.5,\;1.5].
% \]
% Отсюда оценка от \(c_1\):
% \[
% f_2(X)\big|_{c_1} = 0 + [-4,\;0.73]\cdot[-1.5,\;1.5] = [-6.01,\;6.01].
% \]

% \medskip

% \textbf{Центр 2:} \(c_2\approx 0.97\) (положительная критическая точка)

% Критическая точка найдена из уравнения \(f_2'(x)=5x^4-5+\cos x=0\)

% \[
% c_2 \approx 0.97
% \]
% \[
% f_2(c_2)\approx -3.16
% \]

% Численным исследованием наклонной функции получаем
% \[
% S_2(X,c_2)\approx[-3.52,\;8.04],
% \qquad X-c_2\approx[-2.47,\;0.53].
% \]

% Произведение интервалов и добавление значения в центре даёт оценку от \(c_2\):
% \[
% [-3.52,\;8.04]\cdot[-2.47,\;0.53]\approx[-19.86,\;8.69],
% \]
% \[
% f_2(X)\big|_{c_2} = f_2(c_2) + [-19.86,\;8.69] \approx [-23.03,\;5.52].
% \]

% Пересечение с оценкой от \(c_1\):
% \[
% F_{\mathrm{BCF}}(X) = [-6.01,\;6.01]\;\cap\;[-23.03,\;5.52]=[-6.01,\;5.52].
% \]

% \medskip

% \textbf{Центр 3:} \(c_3 = -c_2\approx -0.97\) (отрицательная критическая точка - симметрично).

% Поскольку \(f_2\) — нечётная функция, всё симметрично: при замене \(c_2\mapsto -c_2\)
% получаем зеркальные интервалы. Непосредственно:
% \[
% f_2(-c_2)\approx 3.16,
% \]
% \[
% S_2(X,-c_2)\approx[-8.04,\;3.52],
% \qquad X-(-c_2)\approx[-0.53,\;2.47],
% \]
% \[
% [-8.04,\;3.52]\cdot[-0.53,\;2.47]\approx[-8.69,\;19.86],
% \]
% \[
% f_2(X)\big|_{-c_2} \approx 3.17 + [-8.69,\;19.86] \approx [-5.52,\;23.03].
% \]

% Пересечение с оценкой от \(c_1\) даёт симметричную BCF:
% \[
% F_{\mathrm{BCF}}(X) = [-6.01,\;6.01]\;\cap\;[-5.52,\;23.03] = [-5.52,\;6.01].
% \]

% \textbf{Сравнение:}

% \[
% \begin{aligned}
% &\text{BCF с }c_2\approx +0.97: \quad F_{\mathrm{BCF}}(X) = [-6.01,\;5.52],\\
% &\text{BCF с }c_2\approx -0.97: \quad F_{\mathrm{BCF}}(X) = [-5.52,\;6.01],\\
% &\text{BCF с }c_2=0 \ (\text{оба центра совпадают}): \quad F_{\mathrm{BCF}}(X) = [-6.01,\;6.01].
% \end{aligned}
% \]

% Полученная оценка (пример для \(c_2\approx 0.97\)):
% \[
% \boxed{F_{\mathrm{BCF}}(X) = [-6.01,\;5.52]}
% \]

\subsubsection{Бицентрированная форма (BCF)}

Бицентрированная среднезначная форма определяется как пересечение двух
дифференциальных (среднезначных) центрированных форм, взятых в специально
подобранных центрах \(c_\ast\) и \(c^\ast\):
\[
f_{\mathrm{bic}}(X):=f_{\mathrm{mv}}(X,c_\ast)\cap f_{\mathrm{mv}}(X,c^\ast).
\]
Выбор центров \(c_\ast\) и \(c^\ast\) даётся теоремой Бауманна; для каждого
координатного индекса \(i\) вводим
\[
p_i := \operatorname{cut}\!\Big(\frac{\operatorname{mid} f'_i(X)}{\operatorname{rad} f'_i(X)},[-1,1]\Big),
\]
где функция \(\operatorname{cut}(\cdot,[-1,1])\) обрезает значение до отрезка \([-1,1]\).
Тогда
\[
(c_\ast)_i = \operatorname{mid} X_i - p_i\cdot\operatorname{rad} X_i,\qquad
(c^\ast)_i = \operatorname{mid} X_i + p_i\cdot\operatorname{rad} X_i.
\]
Рассматриваются два варианта бицентрирования:
\begin{enumerate}
  \item \textbf{BCF$_{\mathrm{mv}}$} — пересечение двух \(\mathrm{fmv}\)-форм,
        т.е. \(f_{\mathrm{mv}}(X,c_\ast)\cap f_{\mathrm{mv}}(X,c^\ast)\);
  \item \textbf{BCF$_{\mathrm{sl}}$} — пересечение двух наклонных форм
        (используем \(\mathrm{f_{sl}}\) в центрах \(c_\ast,c^\ast\)).
\end{enumerate}

\paragraph{\textbf{Для \(f_1(x)=x^3-3x^2+2,\; X=[0,3]\)}} 
\begin{itemize}
  \item Производная: \(f_1'(x)=3x^2-6x\). По отрезку \(X\) имеем
    \[
      f_1'(X)=[-3,\,9],\qquad
      \operatorname{mid}f_1'(X)=3,\quad \operatorname{rad}f_1'(X)=6.
    \]
    Поэтому
    \[
      p = \operatorname{cut}\!\Big(\frac{3}{6},[-1,1]\Big)=0.5.
    \]
    Средина и радиус аргумента: \(\operatorname{mid}X=1.5,\;\operatorname{rad}X=1.5\).
    Отсюда
    \[
      c_\ast = 1.5-0.5\cdot1.5=0.75,\qquad c^\ast=1.5+0.5\cdot1.5=2.25.
    \]
  \item \(\mathrm{f_{mv}}\)-оценки (\(f_{\mathrm{mv}}(X,c)=f(c)+f'(X)\cdot(X-c)\)):
    \[
      f_{\mathrm{mv}}(X,c_\ast)\approx[-6.01,\;20.98],
    \]
    \[
      f_{\mathrm{mv}}(X,c^\ast)\approx[-22.04,\;4.95].
    \]
    Пересечение даёт
    \[
      \boxed{\,F_{\mathrm{BCF}}^{\mathrm{mv}}(X) \approx [-6.01,\;4.95]\;}
    \]
    полуширина \( \mathrm{rad}\approx 5.48\).
  \item Наклонные формы в тех же центрах (BCF$_{\mathrm{sl}}$):
    вычисляя наклоны \(\frac{f(x)-f(c)}{x-c}\) на \(X\) относительно каждого центра,
    получаем интервалы
    \[
      f_{\mathrm{sl}}(X,c_\ast)\approx[-5.91,\;2.95],
    \]
    \[
      f_{\mathrm{sl}}(X,c^\ast)\approx[-13.18,\;2.31],
    \]
    их пересечение
    \[
      \boxed{\,F_{\mathrm{BCF}}^{\mathrm{sl}}(X)\approx[-5.91,\;2.31]\;}
    \]
    полуширина \( \mathrm{rad}\approx 4.11\).
\end{itemize}

\paragraph{\textbf{Для \(f_2(x)=x^5-5x+\sin x,\; X=[-1.5,1.5]\)}}
\begin{itemize}
  \item Производная: \(f_2'(x)=5x^4-5+\cos x\).
    \[
      f_2'(X)\approx[-4.01,\;20.38],
    \]
    \[
      \operatorname{mid}f_2'(X)\approx 8.18,\quad
      \operatorname{rad}f_2'(X)\approx 12.19.
    \]
    Тогда
    \[
      p = \operatorname{cut}\!\Big(\frac{8.18}{12.19},[-1,1]\Big)\approx 0.67.
    \]
    Средина и радиус аргумента: \(\operatorname{mid}X=0,\;\operatorname{rad}X=1.5\).
    Отсюда
    \[
      c_\ast \approx -1.01,\qquad c^\ast\approx 1.01.
    \]
  \item \(\mathrm{f_{mv}}\)-оценки в этих центрах (\(f_{\mathrm{mv}}(X,c)=f(c)+f'(X)\cdot(X-c)\)):
    \[
      f_{\mathrm{mv}}(X,c_\ast)\approx[-6.91,\;54.24],
    \]
    \[
      f_{\mathrm{mv}}(X,c^\ast)\approx[-54.24,\;6.91],
    \]
    пересечение даёт
    \[
      \boxed{\,F_{\mathrm{BCF}}^{\mathrm{mv}}(X)\approx[-6.91,\;6.91]\;}
    \]
    полуширина \( \mathrm{rad}\approx 6.91\).
  \item Наклонные бицентрированные оценки (BCF$_{\mathrm{sl}}$) в тех же
    центрах:
    \[
      f_{\mathrm{sl}}(X,c_\ast)\approx[-5.45,\;24.72],
    \]
    \[
      f_{\mathrm{sl}}(X,c^\ast)\approx[-24.72,\;5.45]
    \]
    пересечение
    \[
      \boxed{\,F_{\mathrm{BCF}}^{\mathrm{sl}}(X)\approx[-5.45,\;5.45]\;}
    \]
    полуширина \( \mathrm{rad}\approx 5.45\).
\end{itemize}

\subsection{Оценки по Липшицу и полуширины}

Константа Липшица \(L\) — максимум модуля производной на интервале:

\medskip

\textbf{Для $f_1$:}
\[
L_1 = \max_{x\in[0,3]}|f_1'(x)| = 9
\]

\textbf{Для $f_2$:}
\[
L_2 = \max_{x\in[-1.5,1.5]}|f_2'(x)| \approx 20.38
\]

\medskip
Полуширина интервала:
\[
\operatorname{rad}(A)=\frac{\operatorname{upper}(A)-\operatorname{lower}(A)}{2}
\]

\medskip

Теоретическая верхняя оценка полуширины:
\[
\operatorname{rad}(F(X)) \le L\cdot \operatorname{rad}(X).
\]

\textbf{Для $f_1$:}
\[
\operatorname{rad}(F(X)) \le 13.5
\]

\textbf{Для $f_2$:}
\[
\operatorname{rad}(F(X)) \le 30.57
\]

\section{Программная реализация}

Лабораторная работа выполнена на языке Python в Visual Studio Code.

\medskip

Использовались библиотеки:
\begin{enumerate}
\item numpy
\item matplotlib 
\end{enumerate}

\section{Интервальные оценки: сводные таблицы результатов}

\subsection{Таблица: \(f_1(x)=x^3-3x^2+2,\; X=[0,3]\)}
\begin{table}[h!]
\centering
\begin{tabular}{lcccc}
\toprule
Метод & Интервал \(F(X)\) & rad(\(F(X)\)  \\
\midrule
Естественное & \([-25,29]\) & 27 \\
Горнер & \([-25,2]\) & 13.5 \\
Дифф. центр & \([-14.875,12.125]\) & 13.5 \\
Наклонная центр & \([-5.594,2.844]\) & 4.22 \\
Бицентр. накл. & \([-5.91,\;2.31]\) & 4.11 \\
Бицентр. средн. & \([-6.01,\;4.95]\) & 5.48 \\
\bottomrule
\end{tabular}
\caption{Сравнение методов для \(f_1\)}
\end{table}

\subsection{Таблица: \(f_2(x)=x^5-5x+\sin x,\; X=[-1.5,1.5]\)}
\begin{table}[h!]
\centering

\begin{tabular}{lcccc}
\toprule
Метод & Интервал \(F(X)\) & rad(\(F(X)\)  \\
\midrule
Естественное & \([-16.09,16.09]\) & 16.09 \\
Горнер & \([-8.49,8.49]\) & 8.49 \\
Дифф. центр & \([-30.57,30.57]\) & 30.57 \\
Наклонная центр & \([-6.01,6.01]\) & 6.01 \\
Бицентр. накл. & \([-5.45,\;5,45]\) & 5.45 \\
Бицентр. средн. & \([-6.91,\;6.91]\) & 6.91 \\
\bottomrule
\end{tabular}
\caption{Сравнение методов для \(f_2\)}
\end{table}

\noindent Полуширина теоретической оценки по Липшицу использована в сравнении: \(\approx 30.57\).

\newpage
\section{Выводы}
\subsection{Сравнение методов}
\begin{itemize}
\item Натуральное расширение: простой, но часто сильно переоценивает область значений из-за эффекта зависимости (dependency).
\item Схема Горнера: улучшает оценки для полиномов; для функций с тригонометрическими членами эффект зависимости всё ещё заметен.
\item Дифференциальная центрированная форма: точность этого метода зависит от ширины интервала производной; если производная сильно колеблется, метод будет давать неточную оценку.
\item Наклонная центрированная форма: чаще даёт значительное улучшение по сравнению с дифференциальной формой.
\item Бицентрированная форма: наклонная форма дала более узкий интервал, близкий к точному, чем среднезначная:
\begin{itemize}
    \item \textbf{Для $f_1$:} точный интервал $[-2, 2]$, $F_{\mathrm{BCF}}^{\mathrm{sl}}(X)\approx[-5.91,\;2.31]$, $F_{\mathrm{BCF}}^{\mathrm{mv}}(X) \approx [-6.01,\;4.95]$
    \item \textbf{Для $f_2$:} точный интервал $[-3.16, 3.16]$, $F_{\mathrm{BCF}}^{\mathrm{sl}}(X)\approx[-5.45,\;5.45]$, $F_{\mathrm{BCF}}^{\mathrm{mv}}(X)\approx[-6.91,\;6.91]$
\end{itemize}
\end{itemize}
\subsection{Выводы для \(f_2(x)=x^5-5x+\sin x\)}
\begin{itemize}
\item Для \(f_2(x)=x^5-5x+\sin x\) на интервале \([-1.5,1.5]\) форма с наилучшим результатом — бицентрированная наклонная (полуширина 5.45).
\item Натуральные преобразования и простая дифференциальная форма могут давать чрезмерно широкие интервалы при большой вариативности производной.
\item Интервал \([-1.5,1.5]\) обоснован — он покрывает найденные критические точки и точки перегиба.
\end{itemize}
\end{document}
