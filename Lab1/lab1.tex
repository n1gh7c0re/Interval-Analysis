\documentclass[12pt,a4paper]{article}
% Russian support
\usepackage[T2A]{fontenc}
\usepackage[utf8]{inputenc}
\usepackage[russian]{babel}
% Math, URL, layout
\usepackage{amsmath,amssymb,amsthm}
\usepackage{geometry}
\usepackage{hyperref}
\usepackage{microtype}
\usepackage{setspace}
\usepackage{caption}
\geometry{
  left=25mm,
  right=25mm,
  top=25mm,
  bottom=25mm
}
\hypersetup{
  colorlinks=true,
  linkcolor=black,
  urlcolor=blue,
  pdfauthor={Тишковец Сергей},
  pdftitle={Отчёт по лабораторной работе №1 — Интервальный анализ}
}
\begin{document}
\selectlanguage{russian}
\begin{titlepage}
  \begin{center}
    {Санкт-Петербургский политехнический университет Петра Великого\\
    Институт прикладной математики и механики\\[0.5em]
    \textbf{Высшая школа прикладной математики и вычислительной физики}}
   
    \vfill
   
    {\LARGE\bfseries Отчёт по лабораторной работе №1\par}
    \vspace{1em}
    {\Large по дисциплине\\[0.3em] {\bfseries «Интервальный анализ»}}
   
    \vfill
   
    \begin{flushright}
        Выполнил\\
        студент гр. 5030102/20202: Тишковец Сергей
    \end{flushright}
   
   
    \begin{flushright}
        Проверил\\
        Преподаватель: Баженов Александр Николаевич
    \end{flushright}
   
    \vfill
    Санкт-Петербург\\
    2025
  \end{center}
\end{titlepage}
\setcounter{page}{2}
\tableofcontents
\newpage
\section{Постановка задачи}
Дана ИСЛАУ:
\begin{equation}\label{eq:Ax=b}
A x = b,\qquad x=(x_1,x_2,x_3)^{T}
\end{equation}
с матрицей $A$ с заданной серединой (mid) и радиусом (rad).
\begin{equation}\label{eq:midA}
\operatorname{mid}A=
\begin{pmatrix}
0.95 & 1.00\\[0.2em]
1.05 & 1.00\\[0.2em]
1.10 & 1.00
\end{pmatrix}
\end{equation}
Радиус может быть двух видов:
\begin{equation}\label{eq:radA_full}
\operatorname{rad}A=\delta
\begin{pmatrix}
1.00 & 1.00\\[0.2em]
1.00 & 1.00\\[0.2em]
1.00 & 1.00
\end{pmatrix}
\end{equation}
или
\begin{equation}\label{eq:radA_firstcol}
\operatorname{rad}A=\delta
\begin{pmatrix}
1.00 & 0.00\\[0.2em]
1.00 & 0.00\\[0.2em]
1.00 & 0.00
\end{pmatrix}
\end{equation}
Задачи:
\begin{enumerate}
  \item Найти диапазоны значений $\delta$, при которых $0\in\det[A]$:
  \begin{equation}\label{eq:A_interval_full}
  A=
  \begin{pmatrix}
  [0.95-\delta,\,0.95+\delta] & [1.00-\delta,\,1.00+\delta]\\[0.2em]
  [1.05-\delta,\,1.05+\delta] & [1.00-\delta,\,1.00+\delta]\\[0.2em]
  [1.10-\delta,\,1.10+\delta] & [1.00-\delta,\,1.00+\delta]
  \end{pmatrix}
  \end{equation}
  или
  \begin{equation}\label{eq:A_interval_regress}
  A=
  \begin{pmatrix}
  [0.95-\delta,\,0.95+\delta] & 1.00\\[0.2em]
  [1.05-\delta,\,1.05+\delta] & 1.00\\[0.2em]
  [1.10-\delta,\,1.10+\delta] & 1.00
  \end{pmatrix}
  \end{equation}
  \item Для минимального значения радиуса $\min\delta$ найти точечную матрицу $A'$ такую, что
  \[
  \det A' = 0.
  \]
  \item Обсудить смысл факта $\det A' = 0$ для задач 2-ракурсной томографии и линейной регрессии.
\end{enumerate}
\newpage
\section{Теория}
\subsection{Интервальные матрицы}
Интервальная матрица $[A]$ — матрица, каждый элемент которой задан интервалом
\[
[a_{ij}]=[\operatorname{mid}a_{ij}-\operatorname{rad}a_{ij},\ \operatorname{mid}a_{ij}+\operatorname{rad}a_{ij}].
\]
где ${mid}a_{ij}$ - середина интервала, ${rad}a_{ij}$ - радиус интервала. Таким образом:
\[
a_{ij}\in[\operatorname{mid}a_{ij}-\operatorname{rad}a_{ij},\ \operatorname{mid}a_{ij}+\operatorname{rad}a_{ij}]
\]
Интервальные матрицы позволяют учитывать неопределенности в данных, например, погрешности измерений или варьирующиеся параметры системы.
\subsection{Особенность матрицы}
Матрица называется вырожденной (особенной), если её определитель равен нулю:
\[
\det(A)=0.
\]
Для прямоугольной матрицы типа $3\times2$ особенность соответствует линейной зависимости столбцов:
\begin{equation}
    \text{столбцы } c_1,c_2 \text{ коллинеарны} \Longleftrightarrow \exists\lambda:\ c_1=\lambda c_2
\end{equation}
% \subsection{Концепция \emph{collinear columns}}
% Матрица $A'$ считается вырожденной, если её две колонки линейно зависимы:
% \[
% a_i \approx \lambda b_i,\quad i=1,2,3,
% \]
% где $a_i$ и $b_i$ — элементы первой и второй колонок соответственно. Для каждой строки можно определить интервал значений набора $\lambda_i$:
% \[
% \lambda_i \in \left[\frac{a^{(i)}_{\min}}{b^{(i)}_{\max}},\ \frac{a^{(i)}_{\max}}{b^{(i)}_{\min}}\right].
% \]
% Пересечение этих интервалов по всем строкам даёт общий диапазон $\lambda$, в котором колонки могут быть коллинеарны.

\subsection{Метод миноров}
Матрица $3\times2$ имеет ранг меньше 2 (вырождена), если все её $2\times2$-миноры содержат 0 в интервальной оценке. Для пары строк \(i<j\) минор
\[
M_{ij}=a_{i1}a_{j2}-a_{i2}a_{j1}.
\]
Интервальная оценка $M_{ij}$ линейна по $\delta$, и порог $\delta_{ij}$ — минимальное $\delta$, при котором $0\in M_{ij}$. Минимальное $\delta_{\min} = \max_{ij} \delta_{ij}$ обеспечивает, что все миноры одновременно содержат 0.
\subsection{Применение в задачах}
\begin{itemize}
  \item \textbf{TOMO (2-ракурсная томография)}: оба элемента строки варьируются $\pm\delta$: $[a_i-\delta,a_i+\delta]$, $[b_i-\delta,b_i+\delta]$.
  \item \textbf{REGRESS (линейная регрессия)}: только первый элемент строки варьируется $[a_i-\delta,a_i+\delta]$, второй фиксирован $b_i$.
\end{itemize}
\section{Программная реализация}
Лабораторная работа выполнена на Python в VS Code. Использовалась библиотека \texttt{numpy}.
\section{Результаты}
% \subsection{Для 2-ракурсной томографии}
% По результатам вычислений найдено:
% \[
% \forall \delta \ge 0.037038:\quad 0\in\det(A).
% \]
% \subsection{Для линейной регрессии}
% По результатам вычислений найдено:
% \[
% \forall \delta \ge 0.075001:\quad 0\in\det(A).
% \]
% \subsection{Минимальный $\delta$ для вырожденности матрицы}
\subsection{Регрессия}
Матрица:
\[
A =
\begin{pmatrix}
0.95 & 1 \\
1.05 & 1 \\
1.10 & 1
\end{pmatrix}, \quad
\mathrm{rad}A =
\begin{pmatrix}
\delta & 0\\
\delta & 0\\
\delta & 0
\end{pmatrix}.
\]
Вторая колонка фиксирована, $2\times2$-миноры:
\[
M_{ij} = a_{i1} - a_{j1}, \quad i,j = 1,2,3.
\]
Пересечение интервалов первой колонки даёт условие
\[
\max_i(c_i-\delta)\le\min_i(c_i+\delta)\quad\Longrightarrow\quad 1.10-\delta\le 0.95+\delta,
\]
откуда
\[
\boxed{\delta_{\text{регрессия}}=\frac{0.15}{2}=0.075}
\]
Для минимального $\delta$ точечная матрица
\[
A'=
\begin{pmatrix}
1.024999 & 1.00\\[0.2em]
1.024999 & 1.00\\[0.2em]
1.024999 & 1.00
\end{pmatrix},
\qquad \det A' = 0.
\]
\textit{Обоснование.}
Матрица становится вырожденной, когда все строки линейно зависимы, то есть первая координата может принимать одно общее значение. Это эквивалентно непустому пересечению интервалов $[c_i-\delta, c_i+\delta]$, что и даёт приведённое условие.
\subsection{Томография}
Матрица с равными радиусами:
\[
A =
\begin{pmatrix}
0.95 & 1 \\
1.05 & 1 \\
1.10 & 1
\end{pmatrix}, \qquad
\mathrm{rad}A =
\begin{pmatrix}
\delta & \delta\\
\delta & \delta\\
\delta & \delta
\end{pmatrix}.
\]
Обозначения:
\[
c_1=0.95,\quad c_2=1.05,\quad c_3=1.10,\qquad a_{i1}\in[c_i-\delta,c_i+\delta],\ a_{i2}\in[1-\delta,1+\delta].
\]
Для пары строк \(i<j\) минор
\[
M_{ij}=a_{i1}a_{j2}-a_{i2}a_{j1}.
\]
Интервальная оценка
\[
M_{ij}\in\big[(c_i-c_j)-\delta(c_i+c_j+2),\ (c_i-c_j)+\delta(c_i+c_j+2)\big].
\]
Отсюда порог для попадания 0 в интервал минора:
\[
\delta_{ij}=\frac{c_j-c_i}{\,c_i+c_j+2\,}.
\]
Посчитаем для каждой пары:\\
\medskip
\textbf{Минор \(M_{12}\):}
\[
c_1-c_2=-0.10,\qquad c_1+c_2+2=0.95+1.05+2=4.00,
\]
\[
M_{12}\in[-0.10-4.00\delta,\ -0.10+4.00\delta],
\]
\[
{\delta_{12}=\frac{0.10}{4.00}=0.025.}
\]
\medskip
\textbf{Минор \(M_{13}\):}
\[
c_1-c_3=-0.15,\qquad c_1+c_3+2=0.95+1.10+2=4.05,
\]
\[
M_{13}\in[-0.15-4.05\delta,\ -0.15+4.05\delta],
\]
\[
{\delta_{13}=\frac{0.15}{4.05}=\frac{1}{27}\approx0.037037\ldots}
\]
\medskip
\textbf{Минор \(M_{23}\):}
\[
c_2-c_3=-0.05,\qquad c_2+c_3+2=1.05+1.10+2=4.15,
\]
\[
M_{23}\in[-0.05-4.15\delta,\ -0.05+4.15\delta],
\]
\[
{\delta_{23}=\frac{0.05}{4.15}=\frac{5}{415}\approx0.012048\ldots}
\]
\bigskip
Так как
\[
\delta_{23}\approx0.01205,\quad \delta_{12}=0.025,\quad \delta_{13}\approx0.037037,
\]
то наименьшее значение \(\delta\), при котором существует возможность сделать все три минoра содержащими 0, равно наибольшему из порогов:
\[
\boxed{\delta_{\min}=\delta_{13}=\frac{1}{27}\approx0.037037}
\]
Для минимального $\delta$ получена точечная матрица
\[
A'=
\begin{pmatrix}
0.987036 & 0.962964\\[0.2em]
1.037962 & 1.012647\\[0.2em]
1.062962 & 1.037038
\end{pmatrix},
\qquad \det A' = 0.
\]
\section{Выводы}
\subsection{Обсуждение результатов}
Факт $\det A' = 0$ указывает на критический уровень
неопределённости $\delta$, при котором система становится вырожденной.
\begin{itemize}
    \item В задачах томографии это означает возможную нестабильность восстановления: система может иметь бесконечно много решений или быть несовместной.
    \item В линейной регрессии вырожденность матрицы наблюдений делает неприменимым метод наименьших квадратов без регуляризации.
\end{itemize}
Таким образом, $\det A' = 0$ служит индикатором предельного уровня неопределённости, влияющего на корректность и устойчивость решения задачи.

% \subsection{Метод коллинеарных столбцов vs Метод миноров}
% Метод коллинеарных столбцов имеет ряд проблем:
% \begin{itemize}
%   \item Требует строго положительных знаменателей, что не всегда выполняется при широких интервалах $\delta$.
%   \item Переоценивает границы интервалов $\lambda_i$ из-за использования экстремальных значений, игнорируя зависимости между элементами строки.
%   \item Сложно обобщается на матрицы с более чем двумя столбцами, в отличие от метода миноров.
% \end{itemize}
% Метод миноров предпочтительнее, поскольку:
% \begin{itemize}
%     \item Напрямую проверяет условие вырожденности (все миноры содержат 0).
%     \item Не зависит от предположений о $\lambda$ или знаках.
%     \item Дает аналитическое решение $\delta$.
% \end{itemize}

\end{document}